\documentclass[mathpazo]{cicp}
\usepackage{bm}
\usepackage{graphicx}
\usepackage{color}
%\usepackage[sort]{cite}
%\usepackage[sort&compress,authoryear,round]{natbib}

%%%%% author macros %%%%%%%%%
% place your own macros HERE
%%%%% end %%%%%%%%%

\begin{document}
%%%%% title : short title may not be used but TITLE is required.
% \title{TITLE}
% \title[short title]{TITLE}
\title{Efficient GPU implementation of multicomponent and multiphase Lattice Boltzmann models.}

%%%%% author(s) :
% single author:
% \author[name in running head]{AUTHOR\corrauth}
% [name in running head] is NOT OPTIONAL, it is a MUST.
% Use \corrauth to indicate the corresponding author.
% Use \email to provide email address of author.
% \footnote and \thanks are not used in the heading section.
% Another acknowlegments/support of grants, state in Acknowledgments section
% \section*{Acknowledgments}
\author[M.~Januszewski and A.~Kuzmin]{M.~Januszewski\affil{1}\corrauth and A.~Kuzmin\affil{2}}
\address{\affilnum{1}\ Institute of Physics, University of Silesia, 40-007 Katowice, Poland\\
\affilnum{2}\ Department of Mechanical and Manufacturing Engineering,
Schulich School of Engineering,
University of Calgary, 2500 University Drive NW ,Calgary, Alberta, T2N 1N4 Canada
}
\emails{{\tt michalj@gmail.com} (M.~Januszewski),{\tt shurik.kuzmin@gmail.com} (A.~Kuzmin)}

% multiple authors:
% Note the use of \affil and \affilnum to link names and addresses.
% The author for correspondence is marked by \corrauth.
% use \emails to provide email addresses of authors
% e.g. below example has 3 authors, first author is also the corresponding
%      author, author 1 and 3 having the same address.
% \author[Zhang Z R et.~al.]{Zhengru Zhang\affil{1}\comma\corrauth,
%       Author Chan\affil{2}, and Author Zhao\affil{1}}
% \address{\affilnum{1}\ School of Mathematical Sciences,
%          Beijing Normal University,
%          Beijing 100875, P.R. China. \\
%           \affilnum{2}\ Department of Mathematics,
%           Hong Kong Baptist University, Hong Kong SAR}
% \emails{{\tt zhang@email} (Z.~Zhang), {\tt chan@email} (A.~Chan),
%          {\tt zhao@email} (A.~Zhao)}
% \footnote and \thanks are not used in the heading section.
% Another acknowlegments/support of grants, state in Acknowledgments section
% \section*{Acknowledgments}


%%%%% Begin Abstract %%%%%%%%%%%
\begin{abstract}
Graphics Processing Units (GPUs) have been succesfully used to simulate
various systems using the Lattice Boltzmann (LB) method.  The simulations were
usually either a straightforward port of a CPU code (without any
regards to performance considerations), or optimized for single phase
flows.  The implementation of more complicated problems such as
multiphase models or models with Nearest-Neighbours (NN) interaction
has never received any special attention.  The purpose of this paper
is twofold: to illustrate the usefulness of GPUs for multiphase simulations,
and to carefully develop an optimized and efficient algorithm for GPU
fluid flows simulations using LB models with NN terms.
\end{abstract}
%%%%% end %%%%%%%%%%%

%%%%% AMS/PACs/Keywords %%%%%%%%%%%
%\pac{}

\ams{52B10, 65D18, 68U05, 68U07{\color{red} We need to find somewhere classifications.}}
\keywords{Lattice Boltzmann method, GPU, Shan-Chen model, Free-energy model, binary liquids, multiphase flow, CUDA, OpenCL, GPGPU}

%%%% maketitle %%%%%
\maketitle


%%%% Start %%%%%%
\section{Introduction}
\label{sec:introduction}
During the last two decades since its introduction \cite{mcnamara}, the Lattice Boltzmann Method (LBM)
became mature enough to compete with other well-established approaches to computational fluid dynamics (CFD).
Due to its kinetic nature, LBM can be used to simulate not only simple
hydrodynamics but also thermal flows \cite{karlin-minimalmodels,yuan-thermal},
micro-flows \cite{ansumali-small-knudsen}, ferrofluids \cite{rosensweig,kuzmin-aniso}, and
multiphase flows \cite{rothman-color,Shan-chen:extended,swift}. In the presence of complex
physical effects such as turbulence, multi-phase flows, flow of oil and gas through porous
material, flow around surfaces with complex geometries, or blood flow with deformable
particles, the solution of the basic Navier-Stokes equations becomes difficult or
impractical. On the other hand, micro-scale calculations for such systems based
on the Molecular Dynamics approach are computationally extremely demanding.
The lattice-Boltzmann method, which focuses on the meso-scale, may be the best
choice for such complex flow problems, being able to capture the salient details of both macro- and micro-flows.

LB is an explicit numerical method, which poses some restrictions on the size of the time step --
it has to be kept relatively small in order to preserve the numerical stability of the scheme.  Long-time
simulations in large domains can thus quickly become impractical due to the high computational demand.
The usual solution to this problem is to use a cluster to run the simulations.  In the last few
years, an interesting alternative in the form of Graphics Processing Units of commodity video
cards has emerged.  The GPUs can be viewed as small supercomputers due to their high
computational power (e.g. 1 TFLOPS for a Tesla C1060) and ability to execute multiple
independent threads simultaneously.  Their low price makes them an interesting hardware
platform for numerical calculations of many kinds, but it has to be kept in mind that not
all problems are well suited for the specific architecture of the GPU.  The Lattice-Boltzmann
method, due it its inherent parallelism and locality is however a perfect match for this
new hardware environment.

This was quickly noticed by the CFD community and in the last few years the method was
implemented on NVIDIA GPUs and applied to many different problems, such as calculation of
the drag coefficient for the two-dimensional flow through cylinders \cite{tolke-twod} and
for a moving sphere in a three-dimensional channel \cite{tolke-GPU}.  In the majority of these
problems only fully local, single-fluid LB models were used.  Only very recently some
work on more complex models with NN interactions has been done {\color{red} cite Succi's paper},
but using a code ported from a CPU version, and not optimized specifically for the GPU
environment.
%%
% To the authors' best knowledge only simple
% hydrodynamics problems GPUs implementation is thoroughly described in the literature.
%%
% This is actually not true anymore -- see the Succi's PRE paper about LBM on GPUs
The aim of this paper is to present efficient GPU-optimized algorithms for LB models
with NN interaction terms.  Such models are abundant and of enormous practical importance.
They include the Shan-Chen model, free-energy binary fluid models and higher-order
boundary conditions for fluid flows.  In the following sections, we first present
the details of the Shan-Chen and free-energy LB models.  The reader
is then briefly introduced to the main features of the CUDA environment
that we use to program the GPUs, which is followed by a thorough description
of the GPU algorithms and an analysis of their efficiency.  In section \ref{sec:benchmark}, the two
binary fluid models are compared in terms of performance and accuracy.
We conclude the paper with a summary of the main findings.

\section{The Lattice-Boltzmann method}

{\color{red} a general overview of basics of LBM needs to be added here (grid, collision, propagation, etc)}

\label{sec:lbm:binary:liquids}
This work concentrates on two popular LBM multiphase models. Originally introduced
in 1993 \cite{Shan-chen:extended} the model attracted a multitude of researchers
by its simplicity.  The free-energy model \cite{swift} is based on the free-energy
phase transition model \cite{landau}. Both models were successfully applied in simulation
of two-component blood flow \cite{halliday-multicomponent}, flow in microchannels
\cite{pooley-contact}, thermal flow \cite{zhang-thermal}, and studies of break-up
of the liquid droplet \cite{nourgaliev-breakup}. The section describes the
two-dimensional implementation of the Shan-Chen and free-energy models with
the D2Q9 model. 

The lattice Boltzmann equation operates on the rectangular grid representing the physical domain. It utilizes the
probability distribution functions containing the information about the macroscopic variables, such as density,
momentum. LBE consists from two parts: local collision and propagation operated from one node to another in the certain
direction specified by the velocity set.  The lattice Boltzmann implementation in a standard view representing the
collision and propagation steps:
\begin{equation}
\label{standard:implementation}
\begin{aligned}
&f_i^{*}(\bm{x},t)=\omega f_i^{eq}(\bm{x},t)-(1-\omega) f_i(\bm{x},t) + F_i,&&\text{ collision step}\\
&f_i(\bm{x}+\bm{c_i},t+1)=f_i^{*}(\bm{x},t),&&\text{ propagation step}, 
\end{aligned}
\end{equation}
where $F_i$ is the external force population. The equilibrium function population is usually given as \cite{}:
\begin{equation}
f_i^{eq}=w_i \rho \biggl(1+\frac{c_{i\alpha}u_{\alpha}}{c_s^2}+\frac{Q_{i\alpha\beta}u_{\alpha}u_{\beta}}{2 c_s^4}\biggr),
\end{equation}
where for D2Q9
$w_i=\left\{\dfrac{4}{9},\dfrac{1}{9},\dfrac{1}{9},\dfrac{1}{9},\dfrac{1}{9},\dfrac{1}{36},\dfrac{1}{36},\dfrac{1}{36},\dfrac{1}{36}\right\}$
and $Q_{i\alpha\beta}=c_{i\alpha} c_{i\beta} - c_s^2 \delta_{\alpha\beta}$ with the sound speed parameter $c_s^2=\dfrac{1}{3}$. The macroscopic variables are restored on
every time step through the probability function population:
\begin{equation*}
\begin{aligned}
&\rho=\sum_i{f_i}
&\rho u_{\alpha}=\sum_i{f_i c_{i\alpha}}
\end{aligned}
\end{equation*}
The multiphase models can be simulated either through the force population and/or special equilibrium function. The
Shan-Chen model utilizes the force population, but the free-energy model is simulated through the special equilibrium
model. Both models are described in the next sections.

\subsection{Shan-Chen model}
In the Shan-Chen model \cite{Shan-chen:extended} the force at a given node depends
on all local neighbours characteristics:
\begin{equation}  \label{Shan-Chen:Shan-Chen:cont}
\bm{F}(\bm{x})=-G\psi(\rho(\bm{x}))\sum_{i}{w_i \psi(\rho(\bm{x}+\bm{c_i}))\bm{c_{i}}},
\end{equation}
where $\psi$ is the function of local characteristics of a node, and taken
as $\psi(\rho)=1-\exp(-\rho)$. Coefficient $G$ is responsible for the attraction
force between the molecules. Overall, the Shan-Chen force is the approximation of
the force between molecules \cite{kwok,kwok-contact-angle}. The force modifies
the momentum fluxes as:
\begin{equation*}
P_{\alpha\beta}=\biggl(c_s^2\rho+\frac{G}{6}\psi^2+\frac{G}{36}%
|\bm{\nabla}\psi|^2 +\frac{G}{18}\psi\Delta\psi\biggr)\delta_{\alpha\beta}-\frac{G}{%
18}\partial_{\alpha}\psi\partial_{\beta}\psi,
\end{equation*}
where the equation of states is changed from original $P=c_s^2 \rho$ \cite{Succi-book}
to the van-der-Waals type equation of state:
\begin{equation}
P=c_s^2 \rho +\frac{G}{6} \psi(\rho)^2.
\end{equation}
The lattice Boltzmann simulation for the Shan-Chen model is implemented through the standard implementation (\ref{standard:implementation}). In the presence of the external force the macroscopic velocity is changed as $\rho u_{\alpha}=\sum_i{f_i c_{i\alpha}}+\dfrac{F_{\alpha}}{2}$. The force population is defined as \cite{guo}:
\begin{equation}
F_i=w_i\biggl(1-\frac{\omega}{2}\biggr)\biggl(3 (c_{i\alpha}-u_{\alpha}) + 9 (c_{i\beta}u_{\beta})c_{i\alpha}\biggr)F_{\alpha}.
\end{equation}
The Shan-Chen model is one-component multiphase model. The multicomponent modification is presented in the Appendix~\ref{sec:binary_shan_chen}.
\subsection{Free-energy model}
The free-energy or pressure model \cite{swift} allows the incorporation of the any equation of state to the lattice Boltzmann equation. In comparison with the Shan-Chen model, the free-energy model is simulated through the special choice of the equilibrium distribution function. The lattice Boltzmann scheme is implemented in a standard form (\ref{standard:implementation}), but the equilibrium function is defined as:
\begin{equation}
\begin{aligned}
f_i^{eq}=&w_i \biggl(p_0(\rho)-3 k\rho\Delta\rho+\rho \frac{c_{i\alpha}u_{\alpha}}{3}+\rho \frac{9 Q_{i\alpha\beta}u_{\alpha}u_{\beta}}{2}\biggr)\\
&+k (w_{ixx} \partial_x \rho \partial_x \rho + w_{ixy} \partial_x \rho \partial_y \rho + w_{iyy} \partial_y \rho \partial_y \rho),
\end{aligned}
\end{equation}
where $w_{ixx}=\left\{\dfrac{1}{3},\dfrac{1}{3},-\dfrac{1}{6},-\dfrac{1}{6},-\dfrac{1}{24},-\dfrac{1}{24},-\dfrac{1}{24},-\dfrac{1}{24}\right\}$,
$w_{iyy}=\left\{\dfrac{1}{3},\dfrac{1}{3},-\dfrac{1}{6},-\dfrac{1}{6},-\dfrac{1}{24},-\dfrac{1}{24},-\dfrac{1}{24},-\dfrac{1}{24}\right\}$
and $w_{ixy}=\left\{0,0,0,0,\dfrac{1}{4},\dfrac{1}{4},-\dfrac{1}{4},-\dfrac{1}{4}\right\}$. The laplacian and the gradient are taken by the numerical stencils:
\begin{equation}
\Delta=\frac{1}{6}
\begin{pmatrix}
1&4&1\\
4&-20&4\\
1&4&1\\
\end{pmatrix}\text{ and }
\partial_x=\frac{1}{12}
\begin{pmatrix}
-1&0&1\\
-4&0&4\\
-1&0&1\\
\end{pmatrix}.
\end{equation}

We implement in terms of convergence,accuracy and performance this model with the Shan-Chen potential, mainly $p_0=\dfrac{\rho}{3}+G\dfrac{\psi^2(\rho)}{6}$. The binary liquid expansion of the free-energy model is presented in Appendix~\ref{sec:binary_free_energy}


\section{The CUDA Programming Environment}


CUDA is implemented in the video card hardware by organizing the GPU around the
concept of a multiprocessor (MP).  Such a multiprocessor consists of several (8 for
devices of compute capability lower than 2.0) scalar processors (SPs), each of which is
capable of executing a thread in a SIMT (Single Instruction, Multiple Threads) manner.
Each MP also has a limited amount of specialized on-chip memory: a set of 32-bit registers,
a shared memory block, a constant cache and a texture cache.  The registers are local
to the scalar processor, but the other types of memory are shared between all SPs
in a MP, which allows data sharing between threads.

{\color{red} TODO: verify the figures in this paragraph}
Perhaps the most salient feature of the CUDA architecture is the memory hierarchy
with 1-2 orders of magnitude differences between access times at each successive levels.

The slowest kind of memory is the host memory accessible only to the CPU.  This memory,
while potentially very large in size, cannot be accessed directly by the GPU, so in order
for data stored there to be useful during a GPU simulation, it has to be copied to the
global device memory.  Such a data transfer goes over the PCI-e bus, which has a maximum
throughput of about 6~GB/s.
% max 3.2 GB/s one way, measured on a 4xGTX285 system

Next in line is the global device memory of the GPU, which can currently span several
gigabytes and which has a bandwidth of about 100~GB/s.  Global memory accesses are however
high-latency operations, taking several hundred clock cycles of the GPU to complete {\color{red} Cite NVIDIA documentation here}
% 131.9 GB/s for GTX 280, see NVIDIA CUDA Best Practices Guide, v3.0

The fastest kind of memory currently available on GPUs is the shared memory block residing
on MPs.  It is currently limited in size to just 16~kB, but has a bandwidth of ca~1.3~TB/s
and usually a latency no higher than that of a~SP register access.

The above description readily suggests an optimization strategy which we will generally
follow in the next section and which can be summarized as: move as much data as possible
to the fastest kind of memory available and keep it there as long as possible, while minimizing
accesses to the slower kinds of memory.

\section{GPU implementation of LBM}

We will now outline an efficient and general GPU LBM algorithm taking into consideration
the facts from the previous section.

\section{Results}
\label{sec:benchmark}

\section{Conclusion}
%%%% Acknowledgments %%%%%%%%
\section*{Acknowledgments}
A.~Kuzmin wants to thank the Alberta Ingenuity Fund for their financial support.

\appendix
\section{Binary fluid Shan-Chen model}
\label{sec:binary_shan_chen}
{\color{red} Michal, you can add here binary Shan-Chen model. You have more information on it than me.}
\section{Binary liquid multiphase model}
\label{sec:binary_free_energy}
In this section we describe the binary liquid model. Binary model is based on
the free energy model \cite{swift,landau}:
\begin{equation}
\mathfrak{F}=\int{J \mathrm{d}V}=\int{\left(c_s^2\rho\ln\rho+\frac{A}{2}\phi^2+\frac{B}{4}\phi^4 + \frac{k}{2}(\nabla \phi)^2 \right)\mathrm{d}V},
\end{equation}
where coefficients $A$ and $B$ are responsible for the phase separation,
coefficient $k$ is responsible for the surface tension between phases and
$\phi$ is the order parameter. Different phases are corresponding to
extreme values of $\phi$ as $-1$ and $1$.
The lattice Boltzmann model for the binary free-energy model is defined
through two distribution sets:
\begin{equation}
\begin{aligned}
&f_i^{eq}=w_i \rho \biggl(1+\frac{u_{\alpha}c_{i\alpha}}{c_s^2}+\frac{Q_{i\alpha\beta}u_{\alpha}u_{\beta}}{2 c_s^4}\biggr), i=0\div8\\
&g_i^{eq}=w_i(\Gamma \mu + \frac{\phi c_{i\alpha} u_{i\alpha}}{c_s^2}+\phi^m \frac{Q_{i\alpha\beta}u_{\alpha}u_{\beta}}{2 c_s^4}), i\neq0 \\
&g_0^{eq}=\phi-\sum_{i\neq0}{g_i^{eq}}\quad,
\end{aligned}
\end{equation}
where $\Gamma$ is the mobility parameter, chemical potential
$\mu=-A\phi+A\phi^3-k\Delta\phi$, $k$ is related to the surface
tension. The macroscopic parameters are calculated as:
\begin{equation}
\begin{aligned}
\rho=\sum_i{f_i}\\
\rho u_{\alpha}=\sum_i{f_i c_{i\alpha}}\\
\phi=\sum_i{g_i}
\end{aligned}
\end{equation}
The equilibrium distribution functions lead to the macroscopic equations
{\color{red} Give citation here}:
\begin{equation}
\begin{aligned}
&\partial_t \rho+ \partial_{\alpha} \rho u_{\alpha}=0\\
&\rho\left(\partial_t+u_{\beta}\partial_{\beta}\right) u_{\alpha}=\\
&-\partial_{\alpha}P_{\alpha \beta} + \nu\partial_{\beta}\left(\partial_{\alpha}u_{\beta}+\partial_{\beta} u_{\alpha} + \frac{1}{3}\partial_{\gamma} u_{\gamma} \delta_{\alpha\beta}\right)\\
&\partial_t \phi + \partial_{\alpha} \phi u_{\alpha}=D\Delta \mu,
\end{aligned}
\label{binary:fluid:system}
\end{equation}
with the bulk pressure $p_0=\rho c_s^2-\frac{B}{2}\phi^2+\frac{3}{4}B \phi^2$.

{\color{red} Michal, need your help here. Which parameters have you used for simulation. Please check above.}
We take equilibrium distribution function for the binary-liquid gas model. In general, it can be presented in the following form:

%%%% Bibliography  %%%%%%%%%%
\bibliographystyle{plain}
\bibliography{paper}

% \begin{thebibliography}{99}
% \bibitem{Berger}M. J. Berger and P. Collela, Local adaptive mesh refinement
% for shock hydrodynamics,
% J. Comput. Phys., 82 (1989), 62-84.
% \bibitem{deBoor}C. de Boor,  Good Approximation By Splines With Variable Knots II, in Springer Lecture
%  Notes Series 363, Springer-Verlag, Berlin, 1973.
% \bibitem{TanTZ} Z. J. Tan, T. Tang and Z. R. Zhang, A simple moving mesh method for one- and
% two-dimensional phase-field equations, J. Comput. Appl. Math., to appear.
% \bibitem{Toro}E. F. Toro, Riemann Solvers and Numerical Methods for Fluid Dynamics,
% Springer-Verlag Berlin Heidelbert, 1999.
% \end{thebibliography}

\end{document}
