\documentclass[a4paper]{report}
\usepackage{amsmath}
\usepackage{fullpage}
\begin{document}

We are solving the Navier-Stokes equations in the 2D Cartesian coordinate system, assuming the system is driven by an
external acceleration field $g$ (in the Y direction):
\begin{align}
	\rho_i \left( \frac{\partial v_{ix}}{\partial t} + v_{ix} \frac{\partial v_{ix}}{\partial x} + v_{iy} \frac{\partial v_{ix}}{\partial y} \right) &= \mu_i
		\left( \frac{\partial^2 v_{ix}}{\partial x ^2 } + \frac{\partial^2 v_{ix}}{\partial y^2} \right) \\
	\rho_i \left( \frac{\partial v_{iy}}{\partial t} + v_{ix} \frac{\partial v_{iy}}{\partial x} + v_{iy} \frac{\partial v_{iy}}{\partial y} \right) &= \mu_i
		\left( \frac{\partial^2 v_{iy}}{\partial x ^2 } + \frac{\partial^2 v_{iy}}{\partial y^2} \right) + g \rho_i
\end{align}
We assume the flow to be stationary, so the time derivatives are 0.  We also take $v_{ix} = 0$, so the first equation
does not give us any further information about the system, while the second equation reduces to:
\begin{equation}
	- \frac{g \rho_i}{\mu_i} = \frac{\partial^2 v_{iy}}{\partial x^2}
\end{equation}
if we also asuume $\partial v / \partial y = 0$ (the flow is fully developed).  The general solution is:
\begin{equation}
	v_{iy} = \frac{g \rho_i}{2 \mu_i} x^2 + C_i x + D_i
	\label{eq:gensol}
\end{equation}
If we assume the phase interface to be located at $x = 0$, the continuity of $v$ requires $D_1 = D_2 = D$ and continuity of $\mu_i \partial_x v_i$ requires $\mu_1 C_1 = \mu_2 C_2$.

The flow happen in a channel of radius $R$, so no-slip boundary conditions are enforced at $x = -R$ and $x = R$:
\begin{align}
	v_{1y}(-R) &= \frac{g \rho_1}{2 \mu_1} R^2 - C_1 R + D = 0\\
	v_{2y}(R) &= \frac{g \rho_2}{2 \mu_2} R^2 + \frac{\mu_1}{\mu_2} C_1 R + D = 0
	\label{eq:vel}
\end{align}
Multiplying the first equation by $\mu_1 / \mu_2$ and adding the result to the second equation, we get:
\begin{equation}
	- \frac{g}{2 \mu_2} (\rho_1 + \rho_2) R^2 = D \left( 1 + \frac{\mu_1}{\mu_2} \right)
\end{equation}
and:
\begin{equation}
	D = -\frac{g (\rho_1 + \rho_2)}{ 2 (\mu_1 + \mu_2)} R^2
\end{equation}
Inserting that into Eq.~\ref{eq:vel}:
\begin{equation}
	R \left(\frac{g \rho_1}{2 \mu_1} -\frac{g (\rho_1 + \rho_2)}{ 2 (\mu_1 + \mu_2)} \right) = C_1
\end{equation}
which can be simplified to:
\begin{equation}
	C_1 = \frac{g R}{2} \frac{\mu_2 \rho_1 - \mu_1 \rho_2 }{\mu_1 (\mu_1 + \mu_2)}
\end{equation}

If we take the viscosities to be the same $\mu_1 = \mu_2 = \mu$, we get:
\begin{align}
	D &= -\frac{g (\rho_1 + \rho_2)}{4 \mu} R^2 \\
	C_2 = C_1 &= \frac{g (\rho_1 - \rho_2)}{4 \mu} R
\end{align}
The velocity profiles then read:
\begin{align}
	v_{1y}(x) &= \frac{g}{4 \mu} \left(2 x^2 - (\rho_1 - \rho_2) R x - (\rho_1 + \rho_2) R^2\right),\qquad x < 0\\
	v_{2y}(x) &= \frac{g}{4 \mu} \left(2 x^2 + (\rho_1 - \rho_2) R x - (\rho_1 + \rho_2) R^2\right),\qquad x > 0 
\end{align}

\end{document}
